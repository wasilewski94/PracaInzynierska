\chapter{Podsumowanie}
\label{ch:podsumowanie}


Cel pracy został osiągnięty, stworzono projekt i oprogramowanie sieciowego systemu akwizycji danych. Założenie optymalizacji kosztu zostało spełnione, układ jest dużo tańszy od rozwiązań komercyjnych, które najczęściej wymagają wykupienia licencji na oprogramowanie pozwalające sterować pomiarami. Główną część systemu - Raspberry Pi w wersji Zero W z modułem Wifi i Bluetooth 4.1 można nabyć za około 50zł, jednak zastąpienie przetestowanej wersji Raspberry Pi 3 wersją ZeroW wymagałoby wykonania dodatkowych testów, gdyż różnice parametrów sprzętowych tych dwóch płytek są znaczące i mogą wprowadzać ograniczenia możliwości systemu.

Opracowano sterownik przetwornika analogowo-cyfrowego podłączonego do komputera za pomocą interfejsu SPI i przetestowano pomyślnie jego działanie.

Sterowniki pozostałych urządzeń obsługiwanych po magistrali I$^2$C nie zostały opracowane.
W trakcie wykonywania pracy zdecydowano, że urządzenia podłączane przez magistralę I$^2$C zostaną obsłużone za pomocą modułu \ang{i2cdev}, gdyż nie wymagają próbkowania z dużą częstotliwością.
 
Podczas tworzenia oprogramowania kierowano się założeniem akwizycji danych z zachowaniem jak najpełniejszej (w granicach możliwości technicznych) informacji o czasie pomiaru. Podjęto szereg działań, które pomogły w spełnieniu powyższego założenia oraz przetestowano działanie sterownika za pomocą aplikacji testowej. 

 Zrealizowano funkcjonalność sterowania pomiarem przez użytkownika z poziomu aplikacji webowej oraz wysyłanie danych za pomocą protokołu MQTT. Obsłużono czujniki wilgotności i ciśnienia przy użyciu sterownika \ang{i2cdev} oraz przetwornika analogowo-cyfrowego za pomocą sterownika \ang{spidev}. Przetwornik przetestowano również za pomocą własnego sterownika i aplikacji testowej. Pomyślnie przetestowano działanie układów i akwizycję danych. 

Użytkownik komunikuje się z systemem dzięki interfejsowi graficznemu aplikacji w przeglądarce internetowej. Zaletą systemu jest jego łatwa rozszerzalność. Dokonanie niewielkich zmian w oprogramowaniu i możliwość podłączenia urządzeń do magistrali SPI i I$^2$C oraz UART i pozostałych pinów GPIO sprawia, że użytkownik jest w stanie rozszerzać system o nowe funkcjonalności. Ponadto fakt, iż Raspberry Pi jest popularnym minikomputerem dostępnym w niskiej cenie, zapewnia użytkownikowi wsparcie w dokonywaniu ewentualnych zmian w konfiguracji i strukturze systemu akwizycji danych. 

\newpage
Projekt może być użyty do zastosowań akwizycji danych, na przykład jako stacja meteorologiczna z możliwością zdalnego dostępu dzięki komunikacji sieciowej. Rozwiązania zawarte w projekcie umożliwiają podgląd danych w trakcie wykonywania pomiaru oraz pobranie danych przez użytkownika w celu wykonania dokładniejszej analizy. Część pracy zostanie wykorzystana podczas misji balonowej Studenckiego Koła Inżynierii Kosmicznej Politechniki Warszawskiej do zbierania danych z czujników udostępnianych przez łącze radiowe w paśmie radioamatorskim.
