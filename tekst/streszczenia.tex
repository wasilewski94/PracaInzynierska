\newpage
\begin{center}
\large \bf
Raspberry Pi jako zoptymalizowany pod względem kosztu sieciowy system akwizycji danych
\end{center}

\section*{Streszczenie}

Praca zawiera opis projektu i implementacji sieciowego systemu akwizycji danych z wykorzystaniem platformy Raspberry Pi. Głównym celem projektu było stworzenie systemu umożliwiającego użytkownikowi akwizycję danych pomiarowych i zachowanie jak najdokładniejszej, w granicach możliwości technicznych, informacji o czasie zebrania próbki. Podstawowym problemem w trakcie projektu było zapewnienie wykonania krytycznych czasowo zadań w wielowątkowym systemie operacyjnym Linux. W pracy przedstawione są możliwości różnych rozwiązań programowych. Projekt zakładał, że użytkownik będzie w stanie ustawiać parametry pomiarów zdalnie za pomocą aplikacji internetowej. Dodatkowymi założeniami były optymalizacja pod względem kosztu i łatwa rozszerzalność projektu. W piątym rozdziale zawarto opis testów sprzętu oraz testów funkcjonalnych i prezentację wyników. Ostatnią częścią pracy jest podsumowanie wraz z ukazaniem dalszych możliwości rozwoju projektu.

\bigskip
{\noindent\bf Słowa kluczowe:} system akwizycji danych, Raspberry Pi, sieć

%\vskip 2cm
\newpage

\begin{center}
\large \bf
Raspberry Pi as a cost-optized network data aquisition system
\end{center}

\section*{Abstract}
Thesis includes project description of the network data aquisition system using Raspberry Pi platform. The main aim of the project was to build a system which can be used to aquisite data and save as much as possible accurate information about meassurement timestamp. The main issue was to ensure finishing time-critical tasks in multithreading Linux operating system. Thesis describes variety of software solutions to the problem. Project assumed that the user should be able to control the system via Web application. Another assumption was cost optimazation and developability of the project. In the fifth chapter hardware tests, functional tests and results summary are described.
The last part of the thesis is the assumption with capabities for futher development.

\bigskip
{\noindent\bf Keywords:} data aquisition system, Raspberry Pi, netowrk

\vfill