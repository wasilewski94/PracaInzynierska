\chapter{Wstęp}
\label{wstep}

\section{Wprowadzenie}

W dzisiejszych czasach w dobie Internetu i rozwoju Internetu Rzeczy rośnie zapotrzebowanie na wydajne i zoptymalizowane pod względem kosztu sieciowe systemy akwizycji danych. Rozwój technologii pozwolił na miniaturyzację czujników oraz zmniejszenie kosztu urządzeń zbierających dane pomiarowe.

Akwizycja (zbieranie) danych jest pierwszym i bardzo ważnym etapem przetwarzania danych, gdyż jakość systemu akwizycji wpływa na wydajność i dokładność całego systemu pomiarowego. Istotnym aspektem jest, by odpowiednio dobrać parametry techniczne systemu do konkretnego zastosowania, uwzględniając rodzaj transmisji, poziomy napięć logicznych czy wydajność prądową układu zasilania. 
Dodatkowo system sieciowy musi używać odpowiedniego do danego zastosowania protokołu transmisji.  

Główny nacisk pracy dyplomowej był położony na zachowanie jak najdokładniejszej informacji o czasie pomiaru i okresie próbkowania. Dodatkowo system powinien być zoptymalizowany pod względem kosztu. Użycie platformy Raspberry Pi daje dosyć duże możliwości, jednak ma też swoje ograniczenia. Pierwszym i jednocześnie największym sprzętowym ograniczeniem, wynikającym z zastosowania tej platformy jest brak przetwornika analogowo-cyfrowego na płytce. W celu obsługi sygnałów analogowych konieczne jest zastosowanie zewnętrznego układu. Parametry techniczne płytki Raspberry Pi oraz interfejsy wymiany danych jakie posiada pozwalają na podłączenie i pracę wielu układów peryferyjnych jednocześnie.


\section{Cel i zakres pracy}
Celem pracy było wykonanie zoptymalizowanego pod względem kosztu sieciowego systemu akwizycji danych z wykorzystaniem platformy Raspberry Pi. Projekt zakładał stworzenie płytki pomiarowej będącej rozszerzeniem do platformy Raspberry Pi umożliwiającej zbieranie danych z zewnętrznych urządzeń przy pomocy dostępnych interfejsów SPI i I$^2$C. Płytka powinna zawierać układy umożliwiające doprowadzenie sygnałów z zewnętrznych czujników. 
Sieciowy system akwizycji danych może być wykorzystany do stworzenia stacji meteorologicznej z funkcją zdalnego dostępu, możliwością sterowania i wyzwalania pomiarów oraz odbioru danych przez interfejs sieciowy.
\newpage
W zakres pracy wchodziło:
\begin{itemize}
\item Opracowanie sterowników jądra i aplikacji użytkownika umożliwiających obsługę przetworników i czujników pomiarowych podłączonych do komputera przez typowe interfejsy sprzętowe, takie jak I$^2$C, SPI
\item Zapewnienie akwizycji danych z zachowaniem jak najpełniejszej (w granicach możliwości technicznych) informacji o czasie pomiaru, w celu umożliwienia późniejszej analizy off-line danych z różnych źródeł z uwzględnieniem korelacji czasowej.
\item Przygotowanie serwerów i/lub programów klienckich umożliwiających udostępnienie zmierzonych danych przez sieć TCP/IP.
\item Realizacja interfejsu pozwalającego na zdalne sterowanie procesem pomiaru.
\end{itemize}

\bigskip

Podstawowym celem projektu było zaprojektowanie i realizacja systemu umożliwiającego użytkownikowi akwizycję danych pomiarowych, z zachowaniem jak najdokładniejszej informacji o znaczniku czasu w momencie zebrania próbki. W wielozadaniowym systemie operacyjnym zastosowanie aplikacji z przestrzeni użytkownika nie daje pewności co do reżimu czasowego wykonania poleceń zawartych w kodzie programu. W przypadku akwizycji danych z zewnętrznych przetworników podłączonych przez magistralę interfejsu szeregowego (np. \ang{spidev} dla interfejsu SPI) informacja o znaczniku czasu zarejestrowana w aplikacji z przestrzeni użytkownika nie jest dokładna i powoduje powstanie nieregularności okresu próbkowania.
W celu zapisania jak najdokładniejszej informacji o czasie pomiaru należy znacznik czasu pobrania próbki zarejestrować w przestrzeni jądra systemu.

W pracy przedstawione są możliwości różnych rozwiązań programowych. Niski koszt zbudowania systemu oraz łatwość rozszerzenia funkcjonalności powodują, że urządzenie stanowi konkurencję dla systemów komercyjnych. Projekt zakładał, że użytkownik będzie w stanie ustawiać parametry pomiarów zdalnie za pomocą aplikacji w przeglądarce internetowej. Dodatkowymi założeniami były optymalizacja pod względem kosztu i łatwa rozszerzalność systemu. W końcowej części pracy zawarto przebieg testów sprzętu oraz testów funkcjonalnych i podsumowanie wyników. 

\section{Układ pracy}
Praca została podzielona na sześć rozdziałów. Po niniejszym rozdziale wstępnym, w Rozdziale \ref{ch:zalozenia} dokonano przeglądu systemów akwizycji danych, przedstawiono założenia i wymagania postawione dla sieciowego systemu akwizycji danych oraz dla oprogramowania umożliwiającego obsłużenie urządzeń dołączanych do platformy Raspberry Pi.
W Rozdziale \ref{ch:projekt} przedstawiono projekt układu pozwalającego na realizację zaprojektowanego systemu akwizycji danych. Rozdział \ref{ch:soft} zawiera opis różnych rozwiązań programowych.
W Rozdziale \ref{ch:testy} przedstawiony jest przebieg oraz wyniki testów systemu. W Rozdziale \ref{ch:podsumowanie} zawarto podsumowanie projektu.

